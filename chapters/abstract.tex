%%
% The BIThesis Template for Graduate Thesis
%
% Copyright 2020-2023 Yang Yating, BITNP
%
% This work may be distributed and/or modified under the
% conditions of the LaTeX Project Public License, either version 1.3
% of this license or (at your option) any later version.
% The latest version of this license is in
%   http://www.latex-project.org/lppl.txt
% and version 1.3 or later is part of all distributions of LaTeX
% version 2005/12/01 or later.
%
% This work has the LPPL maintenance status `maintained'.
%
% The Current Maintainer of this work is Feng Kaiyu.



%The abstract is an independent and complete short article that should summarize and succinctly reflect the main content of the paper. Including research purpose, research methods, research results and conclusions, etc., especially highlighting the research results and conclusions. The Chinese abstract strives to be concise and accurate in language. The recommended length for a doctoral thesis is 1000-1200 words, and the recommended length for a master's thesis abstract is 500-800 words. References, figures, tables, chemical structural formulas, non-publicly known symbols and terms are not allowed to appear in the abstract. The content of the English abstract and the Chinese abstract should be completely consistent, the grammar and wording should be accurate, and the language should be concise and smooth. The English version of the doctoral dissertation of international students should have a "detailed Chinese abstract" of no less than 3,000 words.)

\begin{abstract}
人类运动的理解在各个领域都至关重要,包括计算机图形学、机器人技术、医疗保健和交互式模拟。然而,传统方法通常难以有效地弥合动作语义和行为理解之间的差距,这是由于表示不足和运动层次结构中固有的不确定性所导致的。本论文提出了创新性的解决方案,以增强对人类运动的理解和生成。

首先,我们深入研究了人类运动理解的复杂性,并引入了一种称为模糊定性运动学(FQK)的新方法。这一开创性的框架将模糊推理和定性推理相结合,以便于对富有表现力和不确定性的运动事实进行建模。通过为各种运动属性创建模糊语言变量、术语和隶属函数,FQK实现了从运动序列中提...
\end{abstract}

\begin{abstractEn}
Understanding human motion is crucial in various fields, including computer graphics, robotics, healthcare, and interactive simulations. However, traditional methods often struggle to effectively bridge the gap between motion semantics and action comprehension due to inadequacies in representations and inherent uncertainties in kinematic hierarchies. This thesis proposes innovative solutions for enhancing the understanding and generation of human motion.

First, we delve into the complexities of human motion comprehension and introduce a novel approach termed Fuzzy Qualitative Kinematics (FQK). This groundbreaking framework combines fuzzy inference and qualitative ...
\end{abstractEn}
