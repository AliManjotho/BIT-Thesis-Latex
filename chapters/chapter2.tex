%%
% The BIThesis Template for Graduate Thesis
%
% Copyright 2020-2023 Yang Yating, BITNP
%
% This work may be distributed and/or modified under the
% conditions of the LaTeX Project Public License, either version 1.3
% of this license or (at your option) any later version.
% The latest version of this license is in
%   http://www.latex-project.org/lppl.txt
% and version 1.3 or later is part of all distributions of LaTeX
% version 2005/12/01 or later.
%
% This work has the LPPL maintenance status `maintained'.
%
% The Current Maintainer of this work is Feng Kaiyu.
%
% Compile with: xelatex -> biber -> xelatex -> xelatex

\chapter{Literature Review}

Human beings exhibit a remarkable ability to plan and execute body movements in response to their intentions and environmental stimuli \cite{dummy}. This intrinsic capability has become a central pursuit within artificial intelligence research, as there is a growing interest in the development of algorithms capable of generating motion patterns that mimic human behavior. This interdisciplinary endeavor has attracted attention from a myriad of research domains, such as computer vision \cite{dummy}, computer graphics \cite{dummy}, multimedia \cite{dummy}, robotics \cite{dummy}, and human-computer interaction \cite{dummy}. The objective of human motion generation is to create natural, lifelike, and diverse human motions, which hold immense potential for application across various fields including film production, video games, augmented and virtual reality (AR/VR), human-robot interaction, and digital human representation.

\section{Motion Data Representation}
Human motion data is effectively represented by sequences of human body poses across the temporal ...

\subsection{Keypoint-based Representation}
In keypoint-based representation, the human body is depicted through a series of keypoints, denoting specific anatomical...

\subsection{Rotation-based Representation}
Another prevalent method for representing human pose involves joint angles, signifying the rotation of body parts or segments ...



\subsection{The SMPL Model}
The Skinned Multi-Person Linear (SMPL) model is parameterized by a set of pose and shape parameters, facilitating the generation of a 3D mesh representing a human body in a specific pose and shape (as illustrated in Figure \ref{fig_1})...




